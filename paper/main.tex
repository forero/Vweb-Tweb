\documentclass[useAMS,usenatbib]{mn2e} 
\usepackage{amsmath} 
\usepackage{amssymb} 
\usepackage{graphics}
\usepackage{graphicx}
\usepackage{epsfig}  
\usepackage{hyperref} 
\def\be{\begin{equation}}
\def\ee{\end{equation}}
\def\ba{\begin{eqnarray}}
\def\ea{\end{eqnarray}}

% To highlight comments 
\usepackage{color}
\definecolor{red}{rgb}{1,0.0,0.0}
\newcommand{\red}{\color{red}}
\definecolor{blue}{rgb}{0.1,0.3,0.9}
\newcommand{\blue}{\color{blue}}

\usepackage[normalem]{ulem}
\definecolor{darkgreen}{rgb}{0.0,0.5,0.0}

\newcommand{\documentname}{paper~}
\newcommand{\LCDM}{$\Lambda$CDM~}
\newcommand{\beq}{\begin{eqnarray}}  
\newcommand{\eeq}{\end{eqnarray}}  
\newcommand{\zz}{$z\sim 3$} 

\newcommand{\apj}{ApJ}  
\newcommand{\apjs}{ApJS}  
\newcommand{\apjl}{ApJL}  
\newcommand{\aj}{AJ}  
\newcommand{\mnras}{MNRAS}  
\newcommand{\mnrassub}{MNRAS accepted}  
\newcommand{\aap}{A\&A}  
\newcommand{\aaps}{A\&AS}  
\newcommand{\araa}{ARA\&A}  
\newcommand{\nat}{Nature}  
\newcommand{\physrep}{PhR}
\newcommand{\pasp}{PASP}    
\newcommand{\pasj}{PASJ}    
\newcommand{\avg}[1]{\langle{#1}\rangle}  
\newcommand{\ly}{{\ifmmode{{\rm Ly}\alpha}\else{Ly$\alpha$}\fi}}
\newcommand{\hMpc}{{\ifmmode{h^{-1}{\rm Mpc}}\else{$h^{-1}$Mpc }\fi}}  
\newcommand{\hGpc}{{\ifmmode{h^{-1}{\rm Gpc}}\else{$h^{-1}$Gpc }\fi}}  
\newcommand{\hmpc}{{\ifmmode{h^{-1}{\rm Mpc}}\else{$h^{-1}$Mpc }\fi}}  
\newcommand{\hkpc}{{\ifmmode{h^{-1}{\rm kpc}}\else{$h^{-1}$kpc }\fi}}
\newcommand{\hMsun}{{\ifmmode{h^{-1}{\rm
        {M_{\odot}}}}\else{$h^{-1}{\rm{M_{\odot}}}$~}\fi}}   
\newcommand{\hmsun}{{\ifmmode{h^{-1}{\rm
        {M_{\odot}}}}\else{$h^{-1}{\rm{M_{\odot}}}$}\fi}}   
\newcommand{\Msun}{{\ifmmode{{\rm {M_{\odot}}}}\else{${\rm{M_{\odot}}}$}\fi}}  
\newcommand{\msun}{{\ifmmode{{\rm {M_{\odot}}}}\else{${\rm{M_{\odot}}}$}\fi}}  
\newcommand{\lya}{{Lyman$\alpha$~}}
\newcommand{\clara}{{\texttt{CLARA}}~}
\newcommand{\rand}{{\ifmmode{{\mathcal{R}}}\else{${\mathcal{R}}$ }\fi}}  
\newcommand{\hs}{{\hspace{1mm}}} 
\newcommand{\muavg}{\vert\langle\cos\theta\rangle\vert}
% definition to produce a "less than or similar to" symbol
\def\lsim{~\rlap{$<$}{\lower 1.0ex\hbox{$\sim$}}}

% definition to produce a "greater than or similar to" symbol
\def\gsim{~\rlap{$>$}{\lower 1.0ex\hbox{$\sim$}}}

\begin{document}

\title[Halo alignments with the cosmic web]{Cosmic web alignments with
  the shape, angular momentum and peculiar velocities of dark matter
  halos}   
\author[J.E. Forero-Romero et al.]{
\parbox[t]{\textwidth}{\raggedright 
  Jaime E. Forero-Romero$^{1}$,
  Sergio Contreras$^{2}$,
  Nelson Padilla$^{2}$ 
}
\vspace*{6pt}\\
$^{1}$Uni A
$^{2}$Uni B
}
\maketitle

\begin{abstract}

We study the alignment of dark matter halos with the cosmic web as described by
the tidal and velocity shear fields. We focus on the alignment of their
shape, angular momentum and peculiar velocities. We use a cosmological N-body
simulation that allows to study dark matter halos spanning almost five orders
of magnitude in mass ($10^{9}$-$10^{14}$) \hMsun over spatial scales
of $(0.5$-$1.0)$\hMpc to define the cosmic web. After varying the
numerical parameters in our experiments we find that the halo shape
and the tidal field web presents the strongest alignment with a signal
that gets stronger as the halo mass increases. For the angular momentum we
find alignment signals only halos more massive than $10^{12}$\hMsun
both in the tidal and velocity shear descriptions. The peculiar
velocities of halos show a strong alignment signal for the tidal web
at all halo masses, while in the velocity shear description the signal
is weaker an only present for halos less massive than
$10^{12}$\hMsun. We also find that the main directions of stronger
tidal tension and velocity shear coincide for halos below
$10^{12}$\hMpc but are perpendicular to each other for halos more
massive than this threshold. 

\end{abstract}
\begin{keywords}
methods: N-body simulations, galaxies: haloes, cosmology: theory, dark
matter, large-scale structure of Universe 
\end{keywords}


\section{Introduction}
\label{sec:introduction}

There is a long observational tradition stuying galactic properties
as a function of its large scale environment. In these situations the environment
definition is based on quantities easily accesible to observations
such as the number density. With the advenment of large
surveys the visibility of the cosmic web is clear. As a consequence,
in order to capture this filamentarity, the environment definition
started to be more complex, involving shear properties of the galaxy
density field or the reconstructuted dark matter density field.

The emergence of this web like structure is a great success of large
scale structure growth models based on gravitational
instability. Using numerical simulations allows the time description
of such structure together with the formation of the virialized dark
matter halos, which in turn should host observable galaxies. The
finding of gas filaments that feed galaxies is another theoretical
hint that strenghten the expected connection between galactic
properties ant their place in the cosmic web.

Using high resolution simulations it has been stablished that halos of
a given mass for earlier in denser regions; concentration, spin and
shape also depend on the local density. In the last decade new
algorithms have implemented cosmic web classifications that go beyond
the local density by defining a location to be a peak, sheet, filament
or void depending on the simmetry properties of the local dark matter
distribution. 

There analytical frameworks, such as tidal torque theory, that
aim at describing how galaxies gain angular momentum and expect a
correlation of this quantity with the cosmic web. The observational
studies that try to measure such correlations actually try to use as a
proxy the observed shape of galaxies. In this respect it is useful to
have a theoretical baseline for the correlations of angular momentum
and shape with the cosmic web. 

In the same spirit of describing the place of galaxies within the
cosmic web, there has been an revival of surveys that measure the
cosmic flow patters in the Local Universe. Assuming the correspondence
between the divergenec of the cosmic flow velocity field and the local
matter overdensity (valid in the linear regime) one could construct
accurate maps of the matter density from peculiar velocities. From
this perspective is interesting to look at the expected alignment of
the peculiar velocities.


There is large tradition of alignment measurements of shape and
angular momentum. The main result of these previous studies is that shape
alignment is a robust measurement regardless of the methods and
simulations. On the other hand, the results for the angular momentum
differ in the degree of the alignment. In this \documentname we review
most of these studyes and offer our own study with complementary
numerical techniques and simulations.  We also present for the first
time in the literature new results for the alignment of peculiar velocities
with large scale structure.


The structure of this \documentname is the following. In \S
\ref{sec:theory} we present the theoretical antecedendts for the
alignment studies we present in this paper. In \S
\ref{sec:nbody} we present the N-body
cosmological simulation and halo catalogs next to describe in
\S\ref{sec:algorithms} the two web-finding algorithms we use. In
\S\ref{sec:results} we present our main results about the alignment
of shape, spin and peculiar velocities with respect to the cosmic
web. We continue with a discussion of these results in
\S\ref{sec:discussion} paying special attention to resolution effects
and the conditions that might drive the strong alignment signals
observed in the precedent section. In \S\ref{sec:conclusions} we
present our conclusions.


\section{Theoretical Considerations: Notation and Precendents}
\label{sec:theory}

There is abundant literature on the issue of shape and angular momentum
aligmnent of dark matter haloe with respect to the cosmic web. In this
paper we focus our attention on results published during the last
decade that have made use of large N-body dark matter only
cosmological simulations. There are many works that have addressed
this problem using observational data from large surveys such as the
Sloan Digital Sky Survey (SDSS), however we choose to narrow our
discussion to simulation based studies which are prone to comparison.

Out of the three alignments that we study in this paper -shape,
angular momentum and peculiar velocity- only the the first two have
received wide attention in the literature, being angular momentum the
most popular with twice the number of studies for shape alignment.

These alignments are often measured from the distribution of the
$\mu=\vert\cos\theta\vert$ where $\theta$ is the angle between the two axes of
interest. This is often directly measuered as the absolute value of
the dot product between the two unit vectors along the directions
being tested, for instance in the case of angular momentum one would compute
$\mu=\vert\hat{j}\cdot\hat{n}\vert$ In the case of shape alignments the major
axis is the chosen direction to compare against the cosmic web. 

For an isotropic ditribution of the vector around the direction defined by
$\hat{n}$ the $\mu$ distribution, ranging between $0$ and $1$, should
be flat and its mean value shoud be
$\langle|mu|\rangle=0.5$. If a distribution is biased towards $1$
($\langle\vert\mu\vert\rangle>0.5$) we call this an statistical
alignment along $\hat{n}$, while in the case of a bias towards $0$
($\langle\vert\mu\vert\rangle<0.5$) we talk about an anti-alignment,
meaning an perpendicular orientations with respecto to the $\hat{n}$
direction. 

\cite{Trowland2013} presented a parameterization for the $\mu$
distribution in the case of angular momentm alignment based on
theoretical considerations by \cite{Lee2005} (Eq. \ref{eq:distro} in
the Appendix A). Under this parameterization it was found a unique
correspondece between the full shape of the $\vert\mu\vert$
distribution and its average. In this paper we use that result to only
present the results for the average $\langle\vert\mu\vert\rangle$. 

Table 1 and Table 2 summarize recent results found in the literature for
shape and angular momentum alignment. The Appendix A includes a detailed
description of the definitions, algorithms and simulations used in
each one of these studies. In these tables the first column describe
the reference, the second column summarizes the web finding method
with a single name, the third scale associates an spatial scale to the
web finding methods, in most cases it corresponds to the grid size or
smoothing scale used to interpolate the underlying matter density or
velocity field; The fourth column indicates along which web element
(filament or wall) was measured the alignment; the fifth colume
indicates the strenght of the alignment/anti-alignment, $++$/$--$
indicate a strong alignement/anti-alignment while $+$/$-$ indicate a
weaker signal; the last columns indicates whether the described signal
is present within a defined range of halo mass.

These results can be summarized in three important points:
\begin{itemize}
\item The halo mass of $1-5\times 10^{12}$\hMsun is a threshold mass between
behaviours of no-alignment, alignment or anti-alignment.
\item Halo shape provides a strong alignment signal along filaments
  and sheets, more so for massive haloes. 
\item Halo spin tends to be oriented perpendicular to filaments and
parallel to sheets, but it is a weaker than shape alignment. 
\end{itemize}



\begin{table}
\begin{tabular}{lllll}
Author & Web Method & Halo Finder & Major Axis & Correlation\\
\end{tabular}
SHAPE
\end{table}

\begin{table*}
\begin{tabular}{cccccc}\hline\hline
Author & Web Method & Spatial Scale & Along &
Alignment & Mass dependence\\\hline

{\bf Forero-Romero et al. (2014)} & T-Web & $0.5-1 \hMpc$ & 
filament &$++$ & $>10^{12}$\hMsun\\
&   & & 
filament & $+$ & $<10^{12}$\hMsun\\

&   & & 
wall & $++$ & $>10^{12}$\hMsun\\

&   & & 
wall & $+$ & $<10^{12}$\hMsun\\\hline

{\bf Forero-Romero et al. (2014)} & Vp-Web & $0.5-1 \hMpc$ & 
filament &$--$ & $>10^{12}$\hMsun\\
&   & & 
filament & none & $<10^{12}$\hMsun\\
&   & & 
wall & $--$ & $>10^{12}$\hMsun\\

&   & & 
wall & none & $<10^{12}$\hMsun\\\hline


\cite{Libeskind2013} & V-Web & $1 \hMpc$ & 
filament &$++$ & $>10^{12}$\hMsun\\
&   & & 
filament &$+$ & $<10^{12}$\hMsun\\
&   & & 
wall & $++$ & all masses\\\hline

\cite{Zhang2009}  & Hessian density field &  $2.1$\hMpc & 
filament & $++$ & $>10^{12}\hMsun$\\

& &  & 
filament & $+$ & $<10^{12}\hMsun$\\\hline

\cite{AragonCalvo2007} & Hessian density field & - &
wall & $++$ & $>10^{12}$\hMsun\\

& & - &
wall & $+$ & $<10^{12}$\hMsun\\

& & - &
filament& $++$ & $>10^{12}$\hMsun\\

& & - &
filament& $+$ & $<10^{12}$\hMsun\\\hline \hline

\end{tabular}\\
\caption{Shape alignment with the cosmic web. Summary of theoretical
  results provided by methods similar to ours.}
\end{table*}


\begin{table*}
\begin{tabular}{cccccc}\hline\hline
Author & Web Method & Spatial Scale& Along &
Alignment & Mass dependence\\
 & & ($\hMpc$)& & & \\\hline

{\bf Forero-Romero et al. (2014)} & T-Web & $0.5-1$ & 
filament &none & $<10^{12}$\hMsun\\

&   & & 
filament & - & $>10^{12}$\hMsun\\

&   & & 
wall & none & $<10^{12}$\hMsun\\

&   & & 
wall & none & $>10^{12}$\hMsun\\\hline

{\bf Forero-Romero et al. (2014)} & Vp-Web & $0.5-1$ & 
filament &none & $<10^{12}$\hMsun\\

&   & & 
filament & none & $>10^{12}$\hMsun\\

&   & & 
wall & none & $<10^{12}$\hMsun\\

&   & & 
wall & + & $>10^{12}$\hMsun\\\hline

\cite{Libeskind2013} & V-Web & $1$ & 
filament &$-$ & $>10^{12}$\hMsun\\

&   & & 
filament &$+$ & $<10^{12}$\hMsun\\

&   & & 
wall & $++$ & all masses\\\hline

\cite{Trowland2013} & Hessian density & $2-5$ & 
filament & $-$ & $> 5\times 10^{12}$\hMsun\\
&   & & 
filament & $+$ & $< 5\times 10^{12}$\hMsun\\\hline

\cite{Codis2012} & Morse Theory \& T-Web & $1-5$ & 
filament & $--$ & $>10^{12.5}$\hMsun \\ 

&   & & 
filament & $++$ & $<10^{12.5}$\hMsun \\ 

& & & 
wall & $++$ & all masses\\\hline

\cite{Zhang2009}  & Hessian density &  $2.1$ & 
filament & $++$ & if anticorrelated with shape\\

& &  & 
filament & $--$ & if correlated with shape\\\hline

\cite{AragonCalvo2007} & Hessian density & - &
wall & $++$ & $>10^{12}$\hMsun\\


& & - &
wall & $+$ & $<10^{12}$\hMsun\\

& & - &
filament& $-$ & $>10^{12}$\hMsun\\


& & - &
filament& $+$ & $<10^{12}$\hMsun\\\hline



\cite{Hahn2007} & Tidal Web & $2.1$ & filament & $-$& none\\

& & &
wall& $+$ & $<10^{12}$\hMsun\\
& &    &
wall & $++$ & $>10^{12}$\hMsun\\\hline \hline

\end{tabular}
\caption{Spin alignment with the cosmic web. Summary of theoretical
  results provided by methods similar to ours.}

\end{table*}



\section{N-body simulation and halo finding}
\label{sec:nbody}

In this \documentname we use the Bolshoi simulation that follows the
non-linear evolution of a dark matter density field on cosmological
scales. The volume is a cubic box with 250\hMpc on a side, the matter
density field is sampled with $2048^3$ particles. The  
cosmological parameters in the simulation correspond to the results
inferred from WMAP5 data \citep{2009ApJS..180..306D}, which are also consistent with the more
recent results of WMAP9 \citep{2013ApJS..208...19H}. These parameters are $\Omega_m=0.27$,
$\Omega_{\Lambda} =0.73$, $\sigma_8=0.82$, $n_s=0.95$ and $h=0.70$ for the
matter density, cosmological constant, normalization of the power
spectrum, the slope in the spectrum of the primordial matter
fluctuation and the dimensionless Hubble constant. With this
conditions the mass of each dark matter particle in the simulation
corresponds to $m_p=1.4\times 10^{8}$\hMsun. A more detailed
description of the simulation can be found in
\citep{2011ApJ...740..102K}. 


In this paper we use groups found with a Friends-Of-Friends (FOF) halo
finder using a linking length of $b=0.17$ times the mean interparticle
separation. This choice translates into halos with a density of $570$
times the mean density at $z=0$. The measurements for the shape,
angular momentum and peculiar velocity are done using the set of
particles in each dark matter halo. The definition we use in this
\documentname for the shape comes from the diagonalization of the
reduced inertia tensor.

\begin{equation}
{\mathcal T}_{lm} = \sum_{i}\frac{x_{i,l}x_{i,m}}{R_i^2}, 
\end{equation}
where $i$ is the particle index in the halo and $l,m$ run over the
three spatial indexes  and $R_i^2 = x_{i,1}^2 + x_{i,2}^2 +
x_{i,3}^2$, where the positions are measured with respect to the
center of mass.

The spin is calculated as
\begin{equation}
\vec{J} = \sum_{i}m_p{R_i}\vec{v}_i, 
\end{equation}
where the velocities are also measured with respect to the center of
mass velocity. Finally the peculiar velocity of a halo is computed as
the center of mass velocity.

The halo data used in this paper is publicly available through the
MultiDark
database\footnote{\texttt{http://www.multidark.org/MultiDark/}} and is
thoroughly described in \cite{2013AN....334..691R}. 



\section{Web Finding Algorithms}
\label{sec:algorithms}

We use two algorithms to define the cosmic web in cosmological N-body
simulations. Both are based on the same algorithmic principle, which
determines locally a symmetric tensor which can be diagonalized to yield
three real eigenvalues $\lambda_1>\lambda_2>\lambda_3$ and their
corresponding eigenvectors ${\bf e}_{1}$, ${\bf e}_2$ and ${\bf
  e}_3$. This allows for a local classification into one of the
following four web types: void, sheet, filament and peak depending
on the number of eigenvalues larger than a given threshold
$\lambda_{th}$ is $3$, $2$, $1$ or $0$, respectively.

We use two different symmetric tensors. The first is the shear tensor,
defined as the Hessian of the gravitational potential, normalized in
such a way as to be dimensionless:

\begin{equation}
T_{\alpha\beta} = \frac{\partial^2\phi}{\partial
  r_{\alpha}\partial r_{\beta}}, 
\end{equation}
%
where $\phi$ is the gravitational potential rescaled by a factor $4\pi
G\bar{\rho}=3/2\Omega_m H_{0}^2$ in such a way that the Poission
equation can be written as $\nabla^{2}\phi  = \delta$, where $\delta$
is the matter overdensity, $\bar{\rho}$ is the average matter density,
$H_{0}$ is the Hubble constant at present time and $\Omega_m$ is the
matter density parameter. A detailed presentation of this algorithm
can be found in \citep{Tweb}.

The second tensor is the velocity shear:

\begin{equation}
\Sigma_{\alpha\beta} = -\frac{1}{2H_{0}}\left(\frac{\partial
  v_{\alpha}}{\partial r_{\beta}}+ \frac{\partial v_{\beta}}{\partial
  r_{\alpha}}\right), 
\end{equation}
%
where $v_{\alpha}$ correspond to the components of the peculiar
comoving velocities. With this definition the trace of the shear
tensor is minus the divergence of the velocity field normalized by the
Hubble constant $-\nabla\cdot {\bf}v /H_{0}$. A detailed description
of this algorithm can be found in \citep{Vweb}.




\subsection{Numerical considerations}

In this paper we compute the cosmic web on cubic grids of two different
resolutions $256^3$ and $512^3$. For the T-Web we interpolate first
the matter density field using a Cloud-In-Cell (CIC) scheme. Then we
smooth using a Gaussian kernel with a spatial variance equal to the
size of one grid cell. This smoothed matter density field is
transformed into Fourier space to solve the Poisson equation and find
the gravitational potential $\phi$. The Hessian is computed a finite
differences method. Finally, the eigenvalues and eigenvectors are
computed on each grid point.

For the V-Web we interpolate first the momentum density field over a
grid using the CIC scheme and then apply a gaussian smoothing with a
spatial variance of one grid cell. We use the matter density field,
which is also CIC interpolated and gaussian smoothed, to normalize the
momentum field. This ratio between the momentum and matter density
field is what we consider as the velocity field to compute the shear
tensor on each grid point. In this case we also compute the
eigenvalues and eigenvectors on each grid point.

\section{Our Numerical Experiments}
In this paper we use the data and the methods described above to
perform three diferent kinds of measurements.


\subsection{Preferential Alignment}

The first measurement is a rough approximation to find out along which
axes are halos aligned. We call this Preferential Alignment (PA). 

We use the fact that for a given vector under study $\hat{a}$ and the
tree eigenvectors the following identity holds  
 
\begin{equation}
(\hat{s}\cdot\hat{e}_1)^2 +(\hat{s}\cdot\hat{e}_2)^2 +(\hat{s}\cdot\hat{e}_3)^2 =1.
\end{equation}
%
Using this we know that all halos can be splitted into three groups:

\begin{enumerate}
\item Halos with $(\hat{s}\cdot\hat{e}_1)^2> (\hat{s}\cdot\hat{e}_2)^2$
  and $(\hat{s}\cdot\hat{e}_1)^2> (\hat{s}\cdot\hat{e}_3)$, which can
  be considered to aligned mostly along $\hat{e}_1$.
\item Halos with $(\hat{s}\cdot\hat{e}_2)^2> (\hat{s}\cdot\hat{e}_1)^2$
  and $(\hat{s}\cdot\hat{e}_2)^2> (\hat{s}\cdot\hat{e}_3)^2$, which can
  be considered to aligned mostly along $\hat{e}_2$.
\item Halos with $(\hat{s}\cdot\hat{e}_3)^2> (\hat{s}\cdot\hat{e}_1)^2$
  and $(\hat{s}\cdot\hat{e}_3)^2> (\hat{s}\cdot\hat{e}_2)^2$, which can
  be considered to aligned mostly along $\hat{e}_3$.
\end{enumerate}

If the halo population does not show any preferential alignment, then
all the halos must be evenly distributed along these three
populations. On the contrary, if there is more than one third of the
halo population in one of these sets, then we can talk about a
preferential alignment along one of the axis. However, this statistic 
does not give a precise answer on the degree of the alignment


\subsection{Average Alignment Angle}

We emphasize that all the quantities that we compute in this paper are
independent of any threshold on the eigenvalues that could be used to make a web
clasification into web types. Instead, we focus on the alignments with
respect to the eigenvectors regardless of the web type. In this
context we recall that the eigenvector $\hat{e}_1$ is vector
perpendicular to the plane defining a sheet and also a filamente and
that $\hat{e}_3$ is the vector that marks the direction of a filament
and lies on the plane of a sheet. Therefore we focus on quantifying
the degree of alignment along these two eigenvectors. 

This experiment complements the results obtained by the PA satistic by
computing the average and standard deviation of
$\vert\langle\hat{s}\cdot\hat{e}_1\rangle\vert$ and
$\vert\langle\hat{s}\cdot\hat{e}_3\rangle\vert$.  We perform this
tests in differet populations split into different mass bins
logarithmically spaced between $1\times 10^{9}$\hMsun and
$1\times10^{14}$\hMsun.  

In a separate test we make the same measurements but this time
splitting the halo sample by other properties such as:
circularity, concentration, local matter density, spin and
triaxiality. In this case we take the upper and lower $30\%$ of the
halos according to each property and measure the strenght of the
alignment by the average value of
$\vert\langle\hat{s}\cdot\hat{e}_1\rangle\vert$ and
$\vert\langle\hat{s}\cdot\hat{e}_3\rangle\vert$. 

\section{Results}
\label{sec:results}


\subsection{Preferential Alignment}

\begin{figure*}
\includegraphics[width=0.8\textwidth]{Fig1.pdf}
\caption{Fraction of halos in a mass bin that show a preferential
  alignment with respect to an eigenvector. Each row presents on of
  the three properties studied in this paper: shape (major axis),
  angular momentum and  peculiar velocity. Left column presents the
  results against the Vweb   and the right the Tweb. Strong colors
  refer to $256^3$ grid resolutions and lighter colors to a $512^3$
  grid. The thick black horizontal line at $0.33$ corresponds to the
  expected fraction for a random vector field.
\label{fig:preferential}}
\end{figure*}

Figure \ref{fig:preferential} presents all the results for the
preferential alignment (PA) summarizing to a good extent the main
results of this paper. 

For shape alignment and the Tweb (upper row, right column) we find a strong
preferential alignment along the third eigenvector $\vec{e}_{3}$. This
signal incresases steadily with mass and is almost independent of the
grid resolution. At high masses between $70\%$ and $100\%$ of the
halos have their major axis aligned along $\vec{e}_{3}$ which means
that they mostly lie along filaments and sheets. 

The shape alignment and the Vweb (upper row, left column) gives a
different perspective. Firstly, there seems to be little evidence
for an alignment for masses below $10^{11}-10^{12}\hMsun$, depending
on the grid resolution. Secondly, the alignment at higher masses goes
along the first eigenvector $\vec{e}_{1}$ meaning that they mostly lie
perpendicular to the filaments and sheets. In the discussion section
we clarify this result that at first might seem puzzling.

For the angular momentum alignment and the Tweb (middle row, right
column) we find that at low masses $<10^{12}\hMsun$ no evidence for
any alignment. At higher masses $>10^{12}\hMsun$ there is weak signal
of preferential alignment along the first and second eigenvectors,
between $35\%$ to $45\%$ of the halos are aligned with respect to
$\vec{e}_1$ and $\vec{e}_2$. Correspondingly, between $10\%$ to $20\%$
of the halos are aligned along respect to $\vec{e}_{3}$. This means
that most of the halos are perpendicular to the filaments and do not
have a clear alignment with respect to walls.

The angular momentum in the Vweb (middle row, left column) presents a
a signal of alignment along the second eigenvector $\hat{e}_{2}$,
between $45\%$ to $60\%$ of the halos are aligned along that
direction. While there is a minority of halos aligned with
$\vec{e}_{1}$.  There a clear change in trends around
$10^{11}\hMsun-10^{12}\hMsun$ depending on the grid resolution;
below that mass range there is no evidence for alignment while at
higher masses appear all the trends we describe. This means that most
of the halos tend to lie along walls but do not have a clear alignment
with respect to filaments.

The peculiar velocities show a weak but consistent alignment along the
third eigenvector $\vec{e}_{3}$ of the Tweb for all masses below
$10^{13.0}\hMsun-10^{13.5}\hMsun$ depending on the grid
resolution. $45\%$ of the halos are aligned while only $25\%$ are
aligned along the first eigenvector $\vec{e}_1$. This suggests that
halos tend to move along filaments and parallel to the walls, except
at higher masses where the alignments get randomized.  In contrast,
the peculiar velocities with respect to the Vweb show the same trend
but weaker and only for low mass halos $<10^{12}$.


In the next subsections we present a complementary account of this
results by presenting quantitative results of the average angle
between vector pairs describing the alignments discussed so far.


\subsection{Shape Alignment}

\begin{figure*}
\includegraphics[width=0.75\textwidth]{Fig2.pdf}
\caption{Median of $\cos\theta$ quantifying the shape alignment for
  the Vweb (left) and the Tweb (right) at two different grid
  resolutions. In the upper (lower) panels the angle $\theta$ is
  measured between the halo major axis and the first (third)
  eigenvector.\label{fig:shape_alignment}} 
\end{figure*}

Figure \ref{fig:shape_alignment} presents the main results for the
angles between the first and third eigenvectors and the major shape
axis as a function of halo mass. 

In the case of the Vweb (left column) we have a clear alignment with
respect to the first eigenvector at high masses $>10^{12}\hMsun$, with
values $\muavg\approx 0.8$ well above the expected value of $0.5$ for
random distribution. With respect to the third eigenvector we measure
an anti-alignment with $\muavg\approx0.3$. For low masses
$<10^{12}\hMsun$ we do not detect any aligmnent signal. This is 
consistent with the PA results of massive halos
perpendicular to filaments and parallel to walls.

The Tweb (right column) show alignment trends starting at masses of
$10^{10}$\hMsun, two orders of magnitude less than the V-web. In this
case we measure an alignment along the third eigenvectors and an
antialignment along the first eigenvector. In the later case at the
highest mases $\muavg\approx 0.8$ while in the former $\muavg\approx
0.2$. This strong alignment/antialignment signal mirrors the
interpretation from the PA results that describe halos lying parallel
both to filaments and walls.


\subsection{Angular Momentum Alignment}

\begin{figure*}
\includegraphics[width=0.75\textwidth]{Fig3.pdf}
\caption{Median of $|\cos\theta|$ quantifying the angular momentum
  alingment for the Vweb (left) and the Tweb (right) for two different
  grid resolutions. In the upper (lower) panels the angle $\theta$ is
  measured between the first (third) eivenvector and the angular
  momentum vector.\label{fig:spin_alignment}} 
\end{figure*}


We now take a look at the angular momentum alignment. Figure
\ref{fig:spin_alignment} shows the results as a function of halo mass
following the same panel distributio as in Figure
\ref{fig:shape_alignment}. In all cases we see that these alignment
trends are weaker than shape alignments. For the Vweb low mass halos
$<10^{12}$\hMsun do not show any preferential alignment with the
cosmic web. Halos more massive than this threshold have their angular
momentum slightly perpendicular to the direction defined by the first
eigenvector and uncorrelated to the third eigenvector. This translates
into a weak tendency for the angular momentum to lie parallel to walls.

In the case of the Tweb, the alignment for low mass halos $<10^{12}$\hMsun 
is also absent. More massive halos present a weak alignment along
first eigen vector and anti-alignment with the third eigenvector. This
provides a quantitative expression of the results derived from the
preferential alignment whereby the angular momentum is weakly
perpendicular to filaments.

\subsection{Peculiar velocity Alignment}

\begin{figure*}
\includegraphics[width=0.75\textwidth]{Fig4.pdf}
\caption{Median of $\vert\cos\theta\vert$ quantifying the peculiar velocity
  alignment with the Vweb (left) and the Tweb (right) for two different
  grid resolutions. In the upper (lower) panels the angle $\theta$ is
  measured between the first (third) eivenvector and a halo's peculiar
  velocity.
\label{fig:velocity_alignment}} 
\end{figure*}

Figure \ref{fig:velocity_alignment} shows the results for peculiar
velocities alignments. In the case of the Vweb, the peculiar
velocities show a weak signal of aligmnent ($\muavg\approx0.55$)
along the third eigenvector for low masses $<10^{12}$\Msun and a weak
antialigment at higher masses. The strengt of the alignment also shows
clear dependency on the grid size used to compute the web.

The Tweb shows a stronger alignment with the third eignenvecor at all
masses with $\muavg\approx0.6$ and an antialignment with the first
eigenvector with $\muavg\approx 0.4$. In contrast to the Vweb results,
these trends remain baiscally unchanged at all masses and grid
resolutions, with only minor change for halos masses $>10^{13}$\hMsun.


In the first place we wish to directly compare the results of the two
algorithms. For the two algorithms we have the information for the
eigenvectors and the eigenvalues on exactly the same positions defined
by the grids. This allows us tocompute the pair-wise allignment
between the eigenvectors in the two web finders. 

We are mainly interested in the alginment of the web with
the halos. Therefore we restrict our analysis to the grid cells that are
occupied by halos. If we had decided to perform this kind of analysis
on all the grid cells, the statistics would be dominated by the void
regions, because the dominate in number the fraction of cells in the
simuation.

\subsection{Interweb Alignment}

Perhaps the most striking result so far is that the two webs give
different results for the alignment of massive halos. This is not
completely unexpected given that the two algorithms are based on
different physical premises to obtain the directions defining the
eigenvectors. However, we investigate the origin of the different
alignment statistics by studying the inter-web alignment. 
 

Figure \ref{fig:interweb} shows the values for $\muavg$ between the
two $\vec{e}_1$ eigenvectors in the Tweb and the Vweb.  The Figure
shows that there is an alignment, $\muavg\approx 1.0$, for low mass
halos and an antialignment, $\muavg\approx 0.2$ for massive. 

The transitional scale is located around $(10^{11.5}-10^{12.5})$\hMsun
depending on the grid resolution. The coarse grid $(256^3)$ shows the
transition at higher masses than the fine grid $(512^3)$.   We also
note that the alignment is weaker in the finer grid, ($\muavg\approx
0.7$) than in the coarser grid ($\muavg\approx1.0$). 

These two facts (alignment at low masses and low grid resolution)
points towards an explanation in terms of the  linear / non-linear
growth of structure. When the alignment is present on linear scales
the divergence of the velocity field is proportional to the
overdensity, i.e. the trace of the shear field is proportional to the
trace of the tidal field. 

On the scale where the halos more massiven than $10^{13}$\hMsun are
located, the relationshop between the velocity shear and the tidal
field changes. There, the fastest momentum-weighted collapse direction
(defined by the V-web) is perpendicular to the direction where the
tidal compression is the highest.  




\begin{figure}
\includegraphics[width=0.40\textwidth]{Fig5.pdf}
\caption{Median of the interweb alignment for the two grid
  resolutions as a function of the dark matter mass corresponding to
  the halos where the measurement was made. The error bars indicate
  the lower and upper quartiles (Figure in color in the web version).
\label{fig:interweb}} 
\end{figure}


\subsection{What drives the alignment}

We wish to understand what other selection criteria on halo
properties can produce a stronger local alignment for the shape, spin
and peculiar velocities. We split the halo population into low and
high mass halos imposing a cut at $M_{\rm halo}=10^{11}$\hMsun. This
allows us to have robust statistics on the high mass end. We have also
computed these results for a cut at $M_{\rm halo}=10^{12}$\hMsun and
checked that the results we report below are not affected by this
change.

For each mass interval we perform cuts in the following properties:
spin, concentration, triaxiallity, circularity  and density. We
measure the web alignments in two sets, each one including the $30\%$
of halos in the lower/higher end of the corresponding property.

{\bf Possible conclusion. Prolateness follow concentration. In the Tweb
Angular momentum alignment is influenced by spin at higher masses. In
the Vweb angular momentum alginment is influenced by all factors. In
the peculiar velocity alginment only spin seemss to have an effect. }

\begin{figure*}
\includegraphics[width=0.90\textwidth,angle=90]{Fig6.pdf}
\caption{The alignment is always taken with respect to the third eigenvector.}
\end{figure*}
\section{Discussion}
\label{sec:discussion}


We do not find a strong signal for the alignment of spin and the
cosmic web. A significant correlation signal has been recently
measured between the spin and the cosmic vorticity
\citep{Libeskind2013b,Laigle2013}. This two facts are consistent with
spin being produced by the non-symmetric part of the shear and tidal
tensors.  




\section{Conclusions}
\label{sec:conclusions}

We have examined the alignment of shape, angular momentum and peculiar
velocity of dark matter halos with respect to the cosmic web. We use
publicly available data from two algorithms implemented on a large
cosmological N-body simulation to study halo populations spanning five
orders of magnitude in mass. The first algorithm uses the tidal field
and the second the velocity shear, both present local results on
scales of $0.5$\hMpc to $1.0$\hMpc.  

We quantify the alignments in two complementary ways. The first one
measures the fraction of halos in a poulation that is preferentially
aligned with either one of the eigenvectors $\vec{e}_1$, $\vec{e}_2$
or $\vec{e}_3$. The second method measures the average value of the
angle between an eigenvector and the quantity of interest. These two
measurements give us a complete picture for the different degrees of
alignment in the cosmic web. 

We find that the strongest alignment occurs for the halo shape with
respect to the Tweb. In this case the halos tend to align with the
third eigenvector, $\vec{e}_3$, meaning that they lie along filaments
and walls. This trend is gets stronger as the halo mass
increases and agrees with all the results published so far. Instead,
for the Vweb, there is only an antialignment for halos more massive
than $10^{12}$\hMsun, a result that is presented here for the first
time.  

A much weaker alignment signal is present for the angular momentum. In
the Tweb only the most massive halos $>10^{12}$\hMsun are antialigned
with respect to the filaments, while for the Vweb the massive halos
are alignment with sheets. These results broadly agree with the
published literature. Nevertheless, in the same publications there is an
alignment/anti-alignment signal reported at lower halo masses
$<10^{12}$\hMsun that we do not detect in our measurements. 

A new result from our study is the alignment for the peculiar
velocities. Here we find a relatively strong signal of alignment along
the direction defined by the third eigenvector and perpendicular to
the first. This signal is very strong in the Tweb for all masses below
$<10^{13}$\hMsun. This can be interpreted as a flow parallel to walls
and filaments. In the case of the Vweb similar signal, albeit weaker,
is present only for the low mass halos $<10^{12}$\hMsun.

The different behaviour for the alignments of massive halos in the
Tweb and the Vweb was tracked to a corresponding anti-alignment
between the eigenvectors in the two web grids for massive halos
$>10^{12}$\hMsun. For low mass halos the directions defined by the two
webs point in the same direction. This trends can be interpreted as
non-linear effects that appear in the two different physical
descriptions for the cosmic web. 

We also performed a simple study to find evidence of halo properties,
other than mass, in driving the alignments. We find that in the case of
shape, halos located in high density regions or with a low value of
the reduced spin parameter tend to show a stronger signal. This trend is
more pronounced in the Tweb than in the Vweb. Concerning angular
momentum we find that the antialignment signal is stronger for halos
with high spin values. In the peculiar velocities we do not find any
effect in the Tweb alignments, and the results for the Vweb show a
wide variations with grid resolution that impedes driving any strong
conclusion. 


\section*{Acknowledgments} 

\section*{Appendix A. Detailed description of previous theoretical results}


The tidal connection between voids in simulations \citep{Platen2008},
the alignment of observed galaxies with the inferred tidal
field\citep{Lee2007,Jones2010}. There are other kind of alignment
statistics based on modifications of the correlation
\citep{Paz2008,Faltenbacher2009} that go beyond a local computation
and therefore are not reviewed in detail in this paper.

\begin{itemize}


\item
\cite{Libeskind2013} %various alignments

They study the shape and angular momentum alignments with the cosmic
web defined by the velocity shear tensor method described in this
paper.  \cite{Libeskind2013} used the Bolshoi simulation and the halo
catalogs we use in this work. Results are reported for three mass bins $M_{\rm
  vir}<10^{11.5}$\hMsun, $11^{11.5}<M_{\rm vir}<12^{12.5}\hMsun$ and
$M_{\rm  vir}>12^{12.5}\hMsun$. The identification of the cosmic web
is done on a grid of $256^3$ with a gaussian smoothing of $\sim
1$\hMpc over the velocity field. The way they compute this smoothed
velocity field differs from our computation. We do it based on the
momentum density field while \cite{Libesking2013} do not take into
account the mass in each cell. 


The aligmnet signal for the angular momentum is weak while the shape
alignmentsignal is very strong. The shape alignment is such that the
eigenvector corresponding to the smallest eigenvalue is aligned with
the major axis. This effect is stronger for more massive halos.  In
other words the major axis of a halo is aligned with a filament, and
lies on the plane that define a sheet. The angular momentum is
anti-aligned with the filament for massive halos and weakly aligned
for low mass halos.  

\item
\cite{Trowland2013}

They used the Millennium Run, which has $2160^3$ particles in a volume
of $500$\hMpc on a side. This corresponds to a particle mass of
$8.6\times 10^{8}$\hMsun. The catalog uses both halos and subhalos
identified with SUBFIND. Only halos with more than 500 particles were
kept to get a robust computation for the spin. The angular momentum is
defined as the sum of the angular momentum of each particle with 
respect to the center of mass.  


The method to define the filamentary structure is based on the
eigenvalues of the hessian of the density.  However, the analysis is
performed on a box of $300$\hMpc on a side. Four different gaussian
smoothing scales are used: $2.0$, $3.0$ and $5.0$\hMpc. 

By fitting the following functional form tothe $\cos(\theta)$ distribution

\begin{equation}
P(\cos\theta) =
(1-c)\sqrt{1+\frac{c}{2}}\left[1-c\left(1-\frac{3}{2}\cos^{2}\theta
  \right)\right]^{-3/2}, 
\label{eq:distro}
\end{equation}
%
they are able to quantify the degree of alignment ($c<0$) or
antialignment ($c>0$).  This parameterization is based on theoretical
expectactions of Tidal Torque Theory (TTT) \citep{Lee2005}. At $z=0$,
the reported value is $c = −0.035 \pm 0.004$, where the uncertainty
was calculated using bootstraping and resampling. 

When the halo sample is divided between low mass and high mass halos
with a transition scale $M_{\star}=5.9\times 10^{12}$\Msun, there is
a weak alignment signal of the angular momentum against the
principal filament axis for halos above that mass, for halos below
that scale there is a weak anti-alignment.

\item 
\cite{Codis2012}

They studies the alignment of the angular momentum dark relative to
the surrounding large scale structure and to the tidal tensor
eigenvalues.

They use a dark matter simulation with $4096^3$ DM particles in a
cubic periodic box of $2000$\hMpc on a side, which corresponds to a
particle mass of $7.7\times 10^9$\Msun. Halos are identified
using a FoF algorithm with a linking length of 0.2 keeping all halos
with more than 40 particles, which sets the minimum halo mass to be
$3\times 10^{11}$\Msun. In their work the particles were sampled on a
$2048^3$ grid and the density field was smoothed with a gaussian
fileter over a scale of $5$\hMpc corresponding to a mass of $1.9\times
10^{14}$. The skeleton was computed over $6^{3}$ overlapping subcubes
and then reconnected. 

The filament finder algorithm is based on Morse theory and defines a
Skeleton to be the set of critical lines joining the maxima of the
density field through saddle points following the gradient. They also
compute the hessian of the potential over the smoothed density field
to get their eigenvectors.

The angular momentum of the halo is defined as $m_{p}\sum_{i}(r_i-\bar{r})\times
(v_i-\bar{v})$ where $\bar{r}$ is the center of mass of the halo and
$\bar{v}$ is the average velocity.

They measure the alignment with each one of the eigenvectors. With
repecto to the minor eigenvector $\vec{e}_{3}$  (the filament direction) there is anti-alignement for masses $M>5\times10^{12}$\Msun and alignment for masses
$<5\times 10^{12}$\Msun; with respect to the intermediate eigenvector $\vec{e}_2$
there is a strong alignment at high masses and no alignment for low
masses; with respect to the major eigenvector $\vec{e}_{1}$
(normal to the wall plane) there is an anti-alignment signal at all
masses. The results from the Skeleton algorithm are in
agreement with the results from the Tidal web.  The transitional mass
is weakly dependent on the smoothing scale, varing between $1-5\times
10^{12}$\hMsun for smoothing scales between $1.0-5.0$\hMpc. 


\item
\citep{Zhang2009}

They studied the angular momentum and shape alignment against
filaments.  They used a dark matter simulation with $1024^3$ DM particles in a
periodic box of $100$ \hMpc on a side. The particle mass is
$6.92\times10^{7}$\hMsun. Dark matter haloes are found using a FOF
algorithm with a linking length of 0.2 times the interparticle
distance. Only halos with more than 500 particles are retained for
further analysis. The angular momentum is measured with positions
repect to the center of mass and the shape is determined using the
non-normalized moment of inertia tensor.

The environment is found using the hessian of the density. The density
field was interpolated over a $1024^3$ grid and then smoothed with a
Gaussian filter of scale $R_{s} = 2.1$\hMpc. There are two methods to
define the direction of a filament. The first method uses the
eigenvalues of the hessian density; they take the filament
direction to be the eigenvector corresponding the single positive
eigenvalue of the hessian. The second method used a line that
connects the two terminal halos in a filament segment.

For the method that uses the eigenvectors, they find that the strenght
of the angular momentum alignment decreases with halo mass. For the
shape they study the alingment of the major axis with the
filament. The find an alignment signal in all mass bins, with an
stronger effect for more massive halos.  

In a final experiment they measure the angular momentum alignment in four
different samples split by the strength of the shape alignment. They
find that halos anti-aligned in shape, show a strong angular momentum
correlation; and a strong angular momentum anti-alignment for halos
with a strong shape alignment. 

\item 
\citep{AragonCalvo2007} %angular momentum alignment


The used the Multi-Scale Morphology Filter to describe th filamentary
structure. The method is based on the Hessian matrix of the density
field, which is computed from the particle distribution using a
Delaunay tesselation field estimatior (DTFE). This allows them to
identify clusters, filaments and walls. 

They used a simulation with $512^3$ particles in a cubic box of $150$\hMpc. The
mass per particle is $2\times 10^{9}$\hMsun.  Halo identitification is
done with the HOP algorithm. They keep halos with more than $50$
particles and less than $5000$, defining a mass range of$1-100\times
10^{11}$\hMsun. The principal axes of each halo are computed from the
non-normalized inertia tensor. The inertia tensor and the angular
momentum are computed with respect to the center of mass of the halo.  

They compute two angles, one with respect to the direction
defining the filaments and the other the walls. Their results make a
distinction between halos of more massive and less massive than
$10^{12}$\hMsun. The angular momentum tends to lie along the plane of
the wall, with a stronger alignment for massive halos. The effect for
filaments is weaker, low mass halos tend to alignt along the filament,
while high mass halos tend to be anti-aligned.  

For the shape they find a very strong alignment along filaments. In
walls the major axis lies along the wall. Both alignments are stronger
for massive halos. 


\item 

\citep{Hahn2007}% various alignments with the Tweb

The used the Tweb method applied on three simulations each of $512^3$
particles, with sizes $L_1=45$\hMpc, $L_2=90$\hMpc and
$L_3=180$\hMpc, this corresponds to particle masses of $4.7, 38.0,
300\times 10^7$\hMsun. Halo identification was done with a FOF
algorithm with 0.2 times the interparticle distance. They considered
halos of at least 300  particles. 

The web is obtained for a grid of $1024^3$ cells, the density field is
obtained with a CIC interpolation and smoothed using a Gaussian
Kernel. All the results correspond to a smoothing scale of
$R_{s}=2.1$\hMpc. 

They report on the angle between the halo angular momentum vector and
the eigenvector corresponding to perpendicular directions to the
sheets and the direction of the filaments. This is divided into two halo
populations according to mass; low mass $5\times 10^{10} - 1.0\times 10^{12}$ and
high mass $>10^{12}$. They find a weake anti-alignment for filaments
and a stronger anti-alignment in the case of the sheets. For the
sheets the effect is stronger for the massive bin. The anti-alignment
along filaments is weak regardless of the mass. They do not report any
other significan statistic, but recognize that they suffer from
small-number statistics in voids. 

\end{itemize}

\bibliographystyle{mn2e}
\bibliography{references} 

\end{document}
